\documentclass{../res/univ-projet}
\usepackage[T1]{fontenc}
\usepackage[utf8]{inputenc}

\usepackage{algorithmic}
\usepackage{algorithm}

\title{\'Etude des méthodes d'authentification à base de mots de passe jetables}
\author{Damien \bsc{PICARD}, Adrien \bsc{SMONDACK}, Claire \bsc{HARDOUIN}, John Yves \bsc{TAYEWO}, Benjamin \bsc{ZIGH}, Gaetan \bsc{FERRY}, Maxime \bsc{MICHOTTE} }

\projet{One Time Project}
\projdesc{\'Etude des syst\`emes de mots de passe jetables}
\filiere{M1SSI}
\logo{../res/logo_univ.png}



\begin{document}
\maketitle

\begin{abstract}
Ce document présente une analyse de différents systèmes d'OTP tels \og{}OTP\fg{} lui même,\og{}HOTP\fg{}, \og{}TOTP\fg{}, \og{}POTP\fg{} 
et \og{}OTPW\fg{}. Toutes les informations présentées et analysées proviennent des \href{http://tools.ietf.org/html/rfc2289}{RFC2289} et 
toutes les RFC basées sur celle-ci soit, \href{http://tools.ietf.org/html/rfc4226}{RFC4226}, \href{http://tools.ietf.org/html/rfc4256}{RFC4256}, 
\href{http://tools.ietf.org/html/rfc6238}{RFC6238}, \href{http://tools.ietf.org/html/rfc4793}{RFC4793}, 
ainsi que de leurs correctifs. Le but de cet article est de déterminer dans quelles conditions ces systèmes sont utilisables, sous quelles conditions 
et avec quelles garanties de sécurité. Ce document réalise également un comparatif entre les systèmes précédemmen cités.
\end{abstract}
\newpage
\tableofcontents
\newpage
\end{document}
 
