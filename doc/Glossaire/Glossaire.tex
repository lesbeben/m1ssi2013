\documentclass{"../../res/univ-projet"}
\usepackage[utf8]{inputenc}
\usepackage[T1]{fontenc}
\usepackage[francais]{babel}
\usepackage{colortbl}
\usepackage{algorithm}
\usepackage{algorithmic}


\logo{../../res/logo_univ.png}
\title{Terminologie et sigles}
\author{\bsc{Michotte} Maxime}
\projet{M1SSI}
\projdesc{Projet de génération d'OTP}
\filiere{M1SSI}
\version{1.1}
\relecteur{\bsc{Zigh} Benjamin}
\signataire{\bsc{Bardet} Magali}
\date{Décembre 2013}

\histentry{1.0}{0?/12/2013}{Version relue et corrigée}
\histentry{0.1}{16/12/2013}{Premier jet}


\begin{document}
\maketitle
%-------------------------------------------------------------------------------
%-------------------------------------------------------------------------------    
\section{Glossaire}
\begin{description}
	\item[Etat de l'art]
  	Dresser un \'{e}tat de l'art dans un domaine consiste à rechercher toutes 
  	les informations existantes concernant ce domaine et à en faire une synthèse.
	\\
	\item[RFC]
  	Les RFC (Request For Comments) sont un ensemble de documents qui font 
  	r\'{e}f\'{e}rence auprès de la Communaut\'{e} Internet et qui d\'{e}crivent, 
  	sp\'{e}cifient, aident à l'impl\'{e}mentation, standardisent et d\'{e}battent de 
  	la majorit\'{e} des normes, standards, technologies et protocoles li\'{e}s à 
  	Internet et aux r\'{e}seaux en g\'{e}n\'{e}ral.
	\\
	\item[Utilisateur]
  	L'utilisateur est une personne physique qui dans notre cas souhaite utiliser 
  	une authentification utilisant les OTP.
	\\
	\item[Mot de passe]
  	Le mot de passe est une m\'{e}thode parmi d'autres pour effectuer une
  	authentification, c'est-à-dire v\'{e}rifier qu'une personne correspond bien à 
  	l'identit\'{e} d\'{e}clar\'{e}e. Il s'agit d'une preuve que l'on possède et que 
  	l'on transmet au service charg\'{e} d'autoriser l'accès. Le mot de passe doit être 
  	tenu secret pour \'{e}viter qu'un tiers non autoris\'{e} puisse acc\'{e}der à 
  	la ressource ou au service.  
  	\\
	\item[OTP]
  	Un Mot de passe unique ou OTP (One-time password) est un mot de passe qui 
  	n'est valable que pour une session ou une transaction.
	\\
	\item[Token]
  	Un token d\'{e}signe dans notre projet un \'{e}l\'{e}ment logiciel ou mat\'{e}riel 
  	tiers servant à la g\'{e}n\'{e}ration d'un OTP.
  	\\
	\item[Serveur]
  	Un serveur est un dispositif informatique mat\'{e}riel ou logiciel qui offre 
  	des services, à diff\'{e}rents clients. Pour notre cas celui-ci permettra 
  	l'association d'un token, la v\'{e}rification de l'OTP g\'{e}n\'{e}r\'{e} par 
  	le token, et l'\'{e}ventuelle re-synchronisation du token.
 	\\ 
	\item[Client]
  	Un client est le logiciel qui envoie des demandes à un serveur.Pour notre 
  	cas celui-ci permettra de communiquer au serveur les OTP g\'{e}n\'{e}r\'{e}s 
  	par le token.
   	\\ 
    \item[Seed] Variable utilis\'ee pour initialiser une s\'equence al\'eatoire.
    \\
    \item[Vérifieur] Un vérifieur est une entité qui pour un mécanisme 
    d'authentification de type défi/réponse propose un défi au prouveur; dans notre
    cas il s'agit du serveur d'authentification.
    \\
    \item[Prouveur] Un prouveur est une entité qui pour un mécanisme 
    d'authentification de type défi/réponse assure au vérifieur qu'il est bien celui qu'il
    prétend être en répondant au défi via la détention d'un secret; dans notre cas,
    il s'agit du client.
\end{description}

%-------------------------------------------------------------------------------
\end{document}
