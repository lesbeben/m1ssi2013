\documentclass[xcolor=table]{beamer}

\mode<presentation> {
\usetheme{Madrid}
}

\usepackage{graphicx}
\usepackage[utf8]{inputenc} 
\usepackage[french]{babel}

\usepackage{multirow}
\usepackage[table]{xcolor}

\title[M1SSI]{Authentification \`{a} base d'OTP}

\institute[Université de Rouen] {
Université de Rouen \\
\medskip
}
\date{\today}

\AtBeginSubsection[] 
{ 
\begin{frame}  
\frametitle{Plan} 
\tableofcontents[currentsubsection,hideothersubsections,subsectionstyle=show/shaded]
\end{frame}
} 

%------------------------------------------
\begin{document}

\begin{frame}
\titlepage
\end{frame}

\begin{frame}
\frametitle{Table des matières}
\tableofcontents
\end{frame}

%------------------------------------------------
\section{Introduction}
%------------------------------------------------

\begin{frame}
\frametitle{L'équipe}
\begin{block}{Le chef}
Adrien \bsc{Smondack}.
\end{block}
\begin{block}{La technique}
Yves \bsc{Adegoloye} et Damien \bsc{Picard}.
\end{block}
\begin{block}{La qualité}
Claire \bsc{Hardouin} et Gaëtan \bsc{Ferry}.
\end{block}
\begin{block}{La MOA}
Maxime \bsc{Michotte} et Benjamin \bsc{Zigh}.
\end{block}
\end{frame}


\begin{frame}
\frametitle{Le projet}

\begin{block}{Clients}
\begin{description}
\item[Magali \bsc{Bardet}:] Enseignante-Chercheuse à l'université de Rouen.\\
\item[Bruno \bsc{Macadré}:] Ingénieur Système à l'université de Rouen.
\end{description}
\end{block}

\begin{block}{Sujet}
Création d'un système d'authentification à base de mots de passe jetables.
\end{block}
\end{frame}




%------------------------------------------------
\section{Concepts fondamentaux}
%------------------------------------------------
\subsection{L'authentification}
\begin{frame}
\frametitle{Authentification}
\begin{block}{Principe de base}
L'utilisation d'une empreinte produite à partir d'un facteur, permettant d'identifier un utilisateur.
\end{block}
\begin{block}{Les types de facteurs}
\begin{description}
\item[Mémoriel:] Mot de passe, phrase de passe...
\item[Matériel:] Clé USB, certificat numérique, carte à puce...
\item[Corporel:] Empreinte digitale, pupille, voix...
\item[Réactionnel:] Signature, challenge...
\end{description}
\end{block}
\end{frame}


\subsection{L'authentification classique}
\begin{frame}
\frametitle{Principe de base}
\includegraphics[scale=0.24]{../graphics/authsimple.png}
\end{frame}

\begin{frame}
\frametitle{Exemple}
\includegraphics[scale=0.21]{../graphics/auth-mdp.png}
\end{frame}



\subsection{L'authentification par OTP}

\begin{frame}
\frametitle{Principe}
\begin{block}{Définition}
    Un OTP est un mot de passe jetable, c'est à dire qu'il satisfait les deux 
  critères suivants:
  \begin{itemize}
    \item Il n'est pas prévisible
    \item Il n'est valide que pour une unique session.
  \end{itemize}
\end{block}

\begin{block}{Utilité}
  \begin{itemize}
    \item Permettre une authentification à 2 facteurs (mi-forte).
    \item Éviter les attaques par rejeu.
  \end{itemize}
\end{block}
\end{frame}

\begin{frame}
\frametitle{Exemples}
\begin{center}
\includegraphics[scale=0.2]{../graphics/googleauth.png}
\hspace{1em}
\includegraphics[scale=0.405]{../graphics/blizzardauth.jpg}
\end{center}


\end{frame}

\subsection{L'authentification cryptographique}
\begin{frame}
\frametitle{Principe}
\end{frame}

\begin{frame}
\frametitle{Utilisation}
\end{frame}

\subsection{Récapitulatif}
\begin{frame}
\frametitle{Récapitulatif}
\end{frame}


\section{La mise en œuvre}

\subsection{Demande du client}

\begin{frame}
\frametitle{Le livrable 1/2}
\begin{block}{État de l'art} 
Un état de l'art comprenant au moins trois protocoles OTP étudiés.
\begin{description}
 \item[OTP] Protocole de base.
 \item[HOTP] Protocole utilisant la fonction HMAC et un compteur de 
  synchronisation.
 \item[TOTP] Protocole reprenant HOTP avec le temps comme compteur.
 \item[EAP-POTP] Architecture pour s'authentifier sur \verb?IEEE 802.1x?. 
 \item[OTPW] Protocole basé sur l'état de la machine et la fonction de hachage RipeMD-160.
\end{description}
\end{block}
\end{frame}

\begin{frame}
\frametitle{Le livrable 2/2}
\begin{block}{Application}
    L'application se ferait sur les protocole qui auraient été retenus après l'étude de 
  l'état de l'art et comporterait pour chaque protocole:
  \begin{itemize}
    \item Un serveur d'authentification
    \item Un client d'authentification
    \item Un token\footnote[1]{Programme permettant à l'utilisateur d'obtenir un 
      OTP pour s'authentifier.}\footnote[2]{Tout les protocoles ne 
      nécessitent pas de tokens.} Android
    \item Un token Linux
  \end{itemize}
\end{block}

\end{frame}

%------------------------------------------------
\subsection{L'état de l'art}
%------------------------------------------------
\begin{frame}
\frametitle{Formation des équipes de travail}
\begin{block}{Équipes de recherche}
  \begin{itemize}
    \item POTP
    \begin{itemize}
      \item Tayewo-John-Yves \bsc{Adegoloye}
      \item Claire \bsc{Hardouin}
    \end{itemize}
    \item HOTP - TOTP
    \begin{itemize}
      \item Gaëtan \bsc{Ferry}
      \item Maxime \bsc{Michotte}
      \item Benjamin \bsc{Zigh}
    \end{itemize}
    \item OTPW - OTP
    \begin{itemize}
      \item Damien \bsc{Picard}
      \item Adrien \bsc{Smondack}
    \end{itemize}
  \end{itemize}
\end{block}
\end{frame}

\begin{frame}
\frametitle{La méthode}
\end{frame}

\begin{frame}
\frametitle{Le bilan: OTP}
\begin{block}{HOTP}

\end{block}
\begin{block}{TOTP}

\end{block}
\end{frame}


\begin{frame}
\frametitle{Le bilan: livrables}
\begin{block}{Module PAM}
\begin{itemize}
\item Facilité d'installation sur de nombreux systèmes.
\item Compatibilité avec les authentifications existantes.
\end{itemize}
\end{block}

\begin{block}{App Android}
\begin{itemize}
\item Grande réserve d'utilisateurs.
\item Facilement transportable.
\end{itemize}
\end{block} 
\end{frame}

\begin{frame}
\begin{center}
\Huge{Démonstration}
\end{center}
\end{frame}


%------------------------------------------------
\subsection{Organisation du développement}
%------------------------------------------------

\begin{frame}
\frametitle{Méthodologie}
\begin{center}
\Huge Agile (adapté)
\normalsize
\begin{block}{Les grands principes}
\begin{itemize}
 \item Test Driven Development (XP).
 \item Intégration continue et Refactoring (XP).
 \item Appropriation collective du code (XP).
 \item Réunions client régulières et adaptabilité.
\end{itemize}
\end{block}
\end{center}

\end{frame}

\begin{frame}
\frametitle{Répartition}
\begin{block}{Équipes de développement}
  \begin{itemize}
    \item Module PAM
    \begin{itemize}
      \item Claire \bsc{Hardouin}
      \item Damien \bsc{Picard}
      \item Adrien \bsc{Smondack}
      \item Maxime \bsc{Michotte}
    \end{itemize}
    \item App Android
    \begin{itemize}
      \item Gaëtan \bsc{Ferry}
      \item Benjamin \bsc{Zigh}
      \item Tayewo-John-Yves \bsc{Adegoloye}
    \end{itemize}
  \end{itemize}
\end{block}
\end{frame}




%------------------------------------------------
\subsection{Aspect technique}
%------------------------------------------------

\begin{frame}
\frametitle{Langages}
\begin{block}{Technologies utilisées}
\begin{itemize}
  \item Pour le module PAM, le C est obligatoire.
  \item Pour le token Android, Java est recommandé. 
\end{itemize}
\end{block}
\end{frame}


\begin{frame}
  \frametitle{Tests et vérifications}
  \begin{itemize}
   \item Procédures de tests écrites au début du développement de chaque composante. (Test Driven Development)
   \item Tests exécutés par les équipes de développement tout au long du processus de création.
   \item  Lorsque des anomalies sont détectées lors des tests, la procédure est la suivante:
    \begin{itemize}
     \item Création d'une note / mémo précisant l'anomalie rencontrée.
     \item Ajout d'une entrée au journal de test précisant la date du test.
     \item Diffusion de la note à l'équipe de développement pour correction.
    \end{itemize}
  \end{itemize}
\end{frame}




\subsection{Le module PAM}
\begin{frame}
\frametitle{PAM: Architecture logicielle}
\begin{figure}
 \includegraphics[scale=0.3]{../graphics/architecture.png} 
 \caption{Schéma de l'architecture logicielle}
\end{figure}

\end{frame}
\begin{frame}
\frametitle{Gestion de secrets}
Développement d'une bibliothèque chargée de la représentation des secrets en mémoire.
\begin{itemize}
  \item Structure en mémoire.
  \item Gestion de ressources mémoire.
  \item Création aléatoire.
  \item Création à partir d'une entrée utilisateur.
  \item Représentation dans différente bases.
\end{itemize}

Permet de faciliter la gestion de la mémoire.
\end{frame}

\begin{frame}
\frametitle{Génération de mots de passe jetable}
développement d'une bibliothèque permettant de générer des mots de passe selon un secret
et un compteur.
\begin{itemize}
\item Permet de générer selon les méthodes HOTP et TOTP.
\item Implémentation respectant la RFC 4226.
\item Repose sur la bibliothèque de gestion des secrets.
\end{itemize}

\end{frame}

\begin{frame}  
\frametitle{Gestion des utilisateurs}
Bibliothèque permettant d'enregistrer les utilisateur dans un fichier.
\begin{itemize}
  \item Structure en mémoire.
  \item Enregistrement.
  \item Lecture.
  \item Gestion ressources en mémoire.
\end{itemize}

Niveau d'abstraction supplémentaire.
\end{frame}

\begin{frame}
\frametitle{L'interface de programmation pour module PAM}
\begin{itemize}
  \item Un module est écrit en langage C.
  \item Doit implémenter une ou plusieurs fonctions correspondant chacune à une fonctionnalités.
  \item Doit être compilé comme une bibliothèque partagée.
  \item L'interface de programmation permet d'interagir avec l'utilisateur.
\end{itemize}

\end{frame}

\begin{frame}
\frametitle{Développement du mécanisme d'authentification}
\begin{itemize}
\item Implémentation de la fonction \verb?pam\_sm\_authenticate? dans le module.
\item Repose sur les bibliothèque de génération de mots de passe jetable et de gestion
des utilisateurs.
\item Implémente les mécaniques de vérification de mots de passe jetable et de resynchronisation.
\end{itemize}
\end{frame}

\begin{frame}
\frametitle{Gestion de la mise à jour des données utilisateurs}
\begin{itemize}
  \item Implémentation de la fonction \verb?pam\_sm\_chauthtok? dans le module.
  \item Repose sur le mécanisme d'authentification et les bibliothèques de gestion
  d'utilisateurs et de secrets.
  \item Facilite la création de secrets pour les utilisateurs.
  \item Accessible via un utilitaire en ligne de commande.
\end{itemize}

\end{frame}



\subsection{L'app Android}
\begin{frame}
\frametitle{Android: Architecture logicielle}
\begin{figure}
 \includegraphics[scale=0.3]{../graphics/architecture.png} 
 \caption{Schéma de l'architecture logicielle}
\end{figure}

\end{frame}

\begin{frame}
\frametitle{Des choses...}
\end{frame}

\subsection{Licence}
\begin{frame}
\frametitle{Les différentes licences}

\end{frame}

\begin{frame}
\frametitle{La licence choisie}

\end{frame}

%------------------------------------------------
\section{Conclusion}
%------------------------------------------------
\begin{frame}  
\frametitle{Plan} 
\tableofcontents[currentsection,hideothersubsections]
\end{frame}

\begin{frame}
\frametitle{Résultat des tests finaux}

\end{frame}

\begin{frame}
\frametitle{Enseignements tirés du projet}

\end{frame}


\begin{frame}
\frametitle{Améliorations possibles}
\end{frame}

\begin{frame}
\frametitle{Conclusion}
\end{frame}




%------------------------------------------------
\end{document}