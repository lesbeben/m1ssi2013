\documentclass{../../res/univ-projet}

\usepackage[utf8]{inputenc}
\usepackage[T1]{fontenc}
\usepackage[francais]{babel}

\title{Plan de Développement}
\author{Adrien \bsc{Smondack}}
\projet{M1SSI}
\projdesc{Projet de génération d'OTP}
\filiere{M1SSI}
\version{1.0}
\relecteur{Benjamin \bsc{Zigh}}
\signataire{Magali \bsc{Bardet}}
\date{Décembre 2013}

\histentry{1.0}{06/12/2013}{Version initiale.}

\begin{document}
\maketitle

%----------------------------------------------------------------------------------------------------------------------------------------------
\section{Contexte du projet}
	\subsection{Origine du projet}
	Ce projet a été mis en place en réponse à un appel d'offre concernant 
	l'implémentation de système d'authentification via OTP par Magali Bardet et 
	Bruno Macadré.
	
	\subsection{Contexte du développement}
	\begin{description} 
		\item [Cadre :] Dans le cadre du projet annuel de la formation M1SSI;
		\item [Période :] Novembre-Avril;
		\item [Contraintes :] Documents techniques et état de l'art rendus pour Janvier.
	\end{description}
	\subsection{Acteurs}
	\begin{description}
		\item [Émetteur :] Bardet, Macadre;
		\item [Soutien technique :] Mme Bardet, Documentation.
		\item [Coach :] M Abdellah-Godard
	\end{description}
	\subsection{Objectifs poursuivis}
	\begin{itemize}
		\item Fournir un état de l'art sur les systèmes existants;
		\item Implémenter une ou plusieurs de ces méthodes;
		\item Garantir une sécurité forte.
	\end{itemize}
	\subsection{Références} 
	cf. references.pdf ??????????????????????????
\newpage

%----------------------------------------------------------------------------------------------------------------------------------------------
\section{Méthodologie de développement}
	\begin{description}
	    \item [Étape 1 :] État de l'art
		    \begin{description}
		        \item [Objectif :] Se familiariser avec les technologie actuelles et faire le points sur celles que l'on devra employer.
		        \item [Activité :] Étudier les RFC.
		        \item [Produit livrable :] Rapport.
		        \item [Responsabilité :] Toute l'équipe.
            \end{description}
	    \item [Étape 2 :] Implémentation des bibliothèques outils
		    \begin{description}
		        \item [Objectif :] Développer la boite à outils.
		        \item [Activité :] Communication client/serveur, fonctions de hachage, HMAC.
		        \item [Produit livrable :] Documentation.
		        \item [Responsabilité :] Toute l'équipe.
		    \end{description}
	    \item [Étape 3 :] Implémentation des modules OTP
		    \begin{description}
		        \item [Objectif :] Avoir des bibliothèques de calcul et de vérification d'OTP.
		        \item [Activité :] Suivre les algorithmes sélectionnés (conformément aux RFC).
		        \item [Produit livrable :] Documentation.
		        \item [Responsabilité :] Par groupes.
		    \end{description}
	    \item [Étape 4 :] Programmer les tokens (si nécessaire, selon le protocole)
		    \begin{description}
		        \item [Objectif :] Fournir un générateur d'OTP.
		        \item [Activité :] Intégration des outils et des modules de calculs d'OTP dans une IHM.
		        \item [Produit livrable :] Les tokens.
		        \item [Responsabilité :]  Par groupes.
		    \end{description}
	    \item [Étape 5 :] Développer un serveur d'authentification et un client
		    \begin{description}
		        \item [Objectif :] Permettre à un utilisateur de s'authentifier sur un service.
		        \item [Activité :] Intégration des modules de vérification d'OTP.
		        \item [Produit livrable :] Serveur d'authentification + client.
		        \item [Responsabilité :] Par groupes.
	        \end{description}
    \end{description}

%----------------------------------------------------------------------------------------------------------------------------------------------
\section{Organisation et responsabilités}
%Définir l’organisation en présentant
%	L’organigramme (schéma)
%	La description des rôles associés à l’organisation
%	La définition des responsabilités et attributions de chaque intervenant

\begin{tabular}{|p{4.8cm}|p{3.2cm}|p{4cm}|p{4cm}|}
	\hline
	{R\^ole} & {Attribution} & {Description} & {Responsabilité}\\
	\hline
    Chef de projet & Adrien \bsc{Smondack} & &\\
    &&&\\
    MOA & Benjamin \bsc{Zigh} & &\\
    Assistant MOA & Maxime \bsc{Michotte} & &\\
     &&&\\
    Responsable technique & Damien \bsc{Picard} & &\\
    Responsable technique Android & Yves \bsc{Adegoloye} & &\\
     &&&\\
    Responsable qualité & Gaetan \bsc{Ferry} & &\\
    Assist. responsable qualité & Claire \bsc{Hardouin} & &\\
    \hline
\end{tabular}

%----------------------------------------------------------------------------------------------------------------------------------------------
\section{Organigramme des t\^aches}
%A partir de l’arborescence produit et du processus de développement choisi, construire et formaliser un Organigramme des Tâches.
%Faire figurer dans l’OT
%	La liste des produits intermédiaires 
%	La liste des tâches élémentaires associées à la réalisation des produits
%	La liste des moyens matériel et/ou logiciels à mettre en place pour réaliser les produits ou exécuter les tâches

%----------------------------------------------------------------------------------------------------------------------------------------------
\section{Évaluation du projet et dimensionnement des moyens}
%Présenter et justifier
%L’évaluation globale de la charge et la répartition par phase 
%Le besoin en moyens et en ressources
%pour la plate-forme de développement (matériels, logiciels et outils)
%pour la plate-forme de tests
Pour rappel, le projet peut être décomposé en 11 tâches distinctes :
\begin{itemize}
  \item Réalisation de l'état de l'art.
  \item Implémentation des outils pour le développement en C.
  \item Implémentation des outils pour le développement en JAVA.
  \item Implémentation des algorithmes pour le calcul des OTP en C.
  \item Implémentation des algorithmes pour le calcul des OTP en JAVA.
  \item Implémentation des client/serveur d'authentification en C.
  \item Développement des tokens en C.
  \item Développement des tokens en JAVA.
  \item Mise en relation des éléments UNIX/UNIX.
  \item Mise en relation des éléments ANDROID/UNIX.
  \item Test finaux.
\end{itemize}


Pour réaliser toutes ces taches, on dispose de 169 jours de complets de travail. Sur ces 169 jours, 7 collaborateurs travaillerons à temps plein sur le projet.
Au total, nous disposont donc d'un total de 1183 jours-hommes de capacité de travail.
Évaluons la charge requise pour chacunes des taches.
L'état de l'art est une étape importante qui nécéssite d'être parfaitement réalisée. Nous estimons la charge de travail nécessaire à sa réalisation à 420 jours-hommes. \\
Les implémentations des outils pour les développement en C et JAVA sont évaluées respectivement à 84 et 63 jours-hommes. \\
Les implémentations des algorithmes pour les calculs d'OTP en C et en JAVA, étapes cruciales à l'avancement du projet, sont évaluées respectivements à 120 et 90 jours-hommmes. \\
Les implémentation des tokens en C et JAVA sont évaluées respectivement à 60 et 90j, l'implémentation Android nécessitant la mise en place d'une interface graphique. \\
Le développement du serveur d'authentification compatible avec toutes les méthodes de génération d'OTP, est évalué à 90 jours-hommes. \\
Une fois les tokens et le serveur mis en place il sera nécessaire de mettre tous les éléments en relation afin de vérifier le bon fonctionnement des communication. Ces taches sont en fait des test. Leurs charge est évaluée à 49 joursxhommes découpés en 21 jours-hommmes pour les tokens UNIX et 28 jours-hommes pour les tokens Android.\\
Enfin l'étape des tests finaux est évaluée à 42 jours-hommes.\\
Ces informations sont résumées dans le tableau suivant :

\begin{tabular}{|l|l|l|}
Tache & Durée & Ressources mobilisées \\ \hline
Etat de l'art & 60 Jours & 7 personnes \\
Outils pour le développement en C & 21 Jours & 4 personnes \\
Outils pour le développement en Java & 21 Jours & 3 personnes \\
Algorithmes pour le calcul des OTP en C & 30 Jours & 4 personnes \\
Algorithmes pour le calcul des OTP en Java & 30 Jours & 3 personnes \\
Client / serveur d'authentification & 45 Jours & 2 personnes \\
Tokens en C & 30 Jours & 2 personnes \\
Tokens en Java & 30 Jours & 3 personnes \\
Mise en relation des éléments UNIX & 7 Jours & 3 personnes \\
Mise en relation des éléments JAVA & 7 Jours & 4 personnes \\
Tests finaux & 6 Jours & 7 personnes \\ \hline
\end{tabular}


%----------------------------------------------------------------------------------------------------------------------------------------------
\section{Planning général}
%Construire et présenter un planning général du projet avec les dates clés du projet et les livraisons prévues.

%N.B. : Un planning détaillé de la première itération sera annexé sous forme de diagramme PERT ou Gantt.

%----------------------------------------------------------------------------------------------------------------------------------------------
\section{Procédés de gestion}
\subsection{Gestion de la documentation}
%Lister les documents à produire pendant le projet en précisant les responsabilités de rédaction, de relecture et d’approbation.
%Définir brièvement  les règles de production et de gestion de la documentation.

\subsection{Gestion des configurations}
%Définir les procédés de gestion à mettre en œuvre pour assurer la maîtrise des configurations pendant le développement  (règles d’identification, organisation des espaces, sauvegardes et archivages, traitement des évolutions, etc.)

%----------------------------------------------------------------------------------------------------------------------------------------------
\section{Revues et points clef}
%Décrire les points clefs prévus dans le planning
%Objectifs
%Calendrier prévisionnel et/ou événements déclenchants
%Objet des vérifications
%Modalités de contrôle
%Intervenants

%----------------------------------------------------------------------------------------------------------------------------------------------
\section{Procédure de suivi et d'avancement}
%Définir les procédures mises en œuvre pendant le projet pour en assurer le suivi
%Suivi interne (processus, modalités, fréquence, etc.)
%Comptes-rendus d’avancement externe (forme, fréquence, destinataires, etc.)
%Réunions prévues

\end{document}
