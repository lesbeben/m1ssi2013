\documentclass{../../res/univ-projet}

\usepackage[utf8]{inputenc}
\usepackage[T1]{fontenc}
\usepackage[francais]{babel}

\logo{../../res/logo_univ.png}
\title{Plan de Développement}
\author{Adrien \bsc{Smondack}, Benjamin \bsc{Zigh}}
\projet{M1SSI}
\projdesc{Projet de génération d'OTP}
\filiere{M1SSI}
\version{1.1}
\relecteur{Gaëtan \bsc{Ferry}}
\signataire{Magali \bsc{Bardet}}
\date{Décembre 2013}

\histentry{1.1}{19/12/2013}{Version relue.}
\histentry{1.0}{19/12/2013}{Version initiale.}
\histentry{0.1}{06/12/2013}{Ébauche.}

\begin{document}
\maketitle

%----------------------------------------------------------------------------------------------------------------------------------------------
\section{Contexte du projet}
	\subsection{Origine du projet}
	Ce projet a été mis en place en réponse à un appel d'offre concernant 
	l'implémentation de système d'authentification via OTP par Magali Bardet et 
	Bruno Macadré.
	
	\subsection{Contexte du développement}
	\begin{description} 
		\item [Cadre :] Dans le cadre du projet annuel de la formation M1SSI;
		\item [Période :] Novembre-Avril;
		\item [Contraintes :] Documents techniques et état de l'art rendus pour Janvier.
	\end{description}

	\subsection{Acteurs}
	\begin{description}
		\item [Émetteur :] Bardet, Macadre;
		\item [Soutien technique :] Mme Bardet, Documentation.
		\item [Coach :] M Abdellah-Godard
	\end{description}

	\subsection{Objectifs poursuivis}
	\begin{itemize}
		\item Fournir un état de l'art sur les systèmes existants;
		\item Implémenter une ou plusieurs de ces méthodes;
		\item Garantir une sécurité forte.
	\end{itemize}

	\subsection{Références} 
	\begin{tabular}{p{1,5cm}>{\raggedright\arraybackslash}p{13cm}}
		{[ANS10]} & {ANSSI. Référentiel général de sécurité. \href{http://www.ssi.gouv.fr/fr/reglementation-ssi/referentiel-general-de-securite}{http://www.ssi.gouv.fr/fr/reglementation-ssi/referentiel-general-de-securite}, 2010.}
		\tabularnewline
		\\
		{[MvOV97]} & {Alfred J. Menezes, Paul C. van Oorschot, and Scott A. Vanstone. Handbook of applied cryptography. CRC Press Series on Discrete Mathematics and its Applications. CRC Press, Boca Raton, FL, 1997. With a foreword by Ronald L.Rivest.}
		\tabularnewline
		\\
		{[RFC98]} & {A One-Time Password System. \href{http://tools.ietf.org/html/rfc2289}{http://tools.ietf.org/html/rfc2289}, 1998.}
		\tabularnewline
		\\
		{[RFC05]} & {HOTP:An HMAC-Based One-Time Password Algorithm \href{http://tools.ietf.org/html/rfc4226}{http://tools.ietf.org/html/rfc4226}, 2005.}
		\tabularnewline
		\\
		{[RFC06]} & {Generic Message Exchange Authentication for the Securer Shell Protocol (SSH).\href{http://tools.ietf.org/html/rfc4256}{http://tools.ietf.org/html/rfc4256}, 2006.}
		\tabularnewline
		\\
		{[RFC07]} & {The EAP Protected One-Time Password Protocol (EAP-POTP). \href{http://tools.ietf.org/html/rfc4793}{http://tools.ietf.org/html/rfc4793}, 2007.}
		\tabularnewline
		\\
		{[RFC11]} & {TOTP: Time-Based One-Time Password Algorithm \href{http://tools.ietf.org/html/rfc6238}{http://tools.ietf.org/html/rfc6238}, 2011.}
		\tabularnewline
		\\
		{[goo]} & {Google Authentificator \href{https://code.google.com/p/google-authenticator/}{https://code.google.com/p/google-authenticator/}.}
		\tabularnewline
		\\
	\end{tabular}
\newpage
%---------------------------------------------------------------------------------------------------------------------------------------------
\section{Méthodologie de développement}
	\begin{description}
	    \item [Étape 1 :] État de l'art
		    \begin{description}
		        \item [Objectif :] Se familiariser avec les technologies actuelles et faire le point sur celles que l'on devra employer.
		        \item [Activité :] Étudier les RFC.
		        \item [Produit livrable :] Rapport.
		        \item [Responsabilité :] Toute l'équipe.
            \end{description}
	    \item [Étape 2 :] Implémentation des bibliothèques outils
		    \begin{description}
		        \item [Objectif :] Développer la boite à outils.
		        \item [Activité :] Communication client/serveur, fonctions de hachage, HMAC.
		        \item [Produit livrable :] Documentation.
		        \item [Responsabilité :] Toute l'équipe.
		    \end{description}
	    \item [Étape 3 :] Implémentation des modules OTP
		    \begin{description}
		        \item [Objectif :] Avoir des bibliothèques de calcul et de vérification d'OTP.
		        \item [Activité :] Suivre les algorithmes sélectionnés (conformément aux RFC).
		        \item [Produit livrable :] Documentation.
		        \item [Responsabilité :] Par groupes.
		    \end{description}
	    \item [Étape 4 :] Programmer les tokens (si nécessaire, selon le protocole)
		    \begin{description}
		        \item [Objectif :] Fournir un générateur d'OTP.
		        \item [Activité :] Intégration des outils et des modules de calculs d'OTP dans une IHM.
		        \item [Produit livrable :] Les tokens.
		        \item [Responsabilité :]  Par groupes.
		    \end{description}
	    \item [Étape 5 :] Développer un serveur d'authentification et un client
		    \begin{description}
		        \item [Objectif :] Permettre à un utilisateur de s'authentifier sur un service.
		        \item [Activité :] Intégration des modules de vérification d'OTP.
		        \item [Produit livrable :] Serveur d'authentification + client.
		        \item [Responsabilité :] Par groupes.
	        \end{description}
    \end{description}
\newpage
%---------------------------------------------------------------------------------------------------------------------------------------------
\section{Organisation et responsabilités}
	\begin{figure}[h]
	\includegraphics[scale=0.5]{organigramme.png}
	\caption{Organigramme des rôles au sein de l'équipe}
	\end{figure}
	\begin{description}
	\item[Chef de Projet :] Chargé de la coordination du projet et de la communication au sein de l'équipe.
	\item[Branche MOA :] Chargée de la communication entre l'équipe et le client.
	\item[Branche Technique :] Chargée de l'organisation du développement.
	\item[Branche Qualité :] Chargée du bon fonctionnement des livrables. 
	\end{description}
\newpage
%---------------------------------------------------------------------------------------------------------------------------------------------
\section{Organigramme des t\^aches}
	L'organisation suivante est déduite de l'état de l'art, qui n'est donc pas inclus.
	\begin{figure}[h]
	\includegraphics[scale=0.5]{taches.png}
	\caption{Organigramme des tâches à accomplir}
	\end{figure}
	\begin{description}
	\item[Hachage :]Implémentation des fonctions de hachages standards recommandées par les RFC (SHA1, Keccac, MD5, RipeMD160).
	\item[Réseau :]Implémentation de bibliothèques pour l'emploi de sockets UNIX.
	\item[HMAC :]Implémentations de HMAC pour différentes fonctions de hachage.
	\item[HOTP, OTPW, TOTP :]Génération des OTP pour les protocoles choisis.
	\item[Token :]Développement de l'interface utilisateur pour la génération d'OTP.
	\item[Client :]Développement de l'interface utilisateur pour la communication avec le serveur.
	\item[Serveur :]Serveur d'authentification compatible avec tous les protocoles choisis.
	\end{description}
\newpage
%---------------------------------------------------------------------------------------------------------------------------------------------
\section{Évaluation du projet et dimensionnement des moyens}
	Pour rappel, le projet peut être décomposé en 11 tâches distinctes :
	\begin{itemize}
	  \item Réalisation de l'état de l'art.
	  \item Implémentation des outils pour le développement en C.
	  \item Implémentation des outils pour le développement en JAVA.
	  \item Implémentation des algorithmes pour le calcul des OTP en C.
	  \item Implémentation des algorithmes pour le calcul des OTP en JAVA.
	  \item Implémentation des client/serveur d'authentification en C.
	  \item Développement des tokens en C.
	  \item Développement des tokens en JAVA.
	  \item Mise en relation des éléments UNIX/UNIX.
	  \item Mise en relation des éléments ANDROID/UNIX.
	  \item Test finaux.
	\end{itemize}


	Pour réaliser toutes ces taches, on dispose de 169 jours complets de travail. Sur ces 169 jours, 7 collaborateurs travaillerons à temps plein sur le projet.
	Au total, nous disposons donc d'un total de 1183 jours-hommes de capacité de travail.
	Évaluons la charge requise pour chacune des taches.
	L'état de l'art est une étape importante qui nécessite d'être parfaitement réalisée. Nous estimons la charge de travail nécessaire à sa réalisation à 420 jours-hommes. \\
	Les implémentations des outils pour les développement en C et JAVA sont évaluées respectivement à 84 et 63 jours-hommes. \\
	Les implémentations des algorithmes pour les calculs d'OTP en C et en JAVA, étapes cruciales à l'avancement du projet, sont évaluées respectivement à 120 et 90 jours-hommes. \\
	Les implémentation des tokens en C et JAVA sont évaluées respectivement à 60 et 90j, l'implémentation Android nécessitant la mise en place d'une interface graphique. \\
	Le développement du serveur d'authentification compatible avec toutes les méthodes de génération d'OTP, est évalué à 90 jours-hommes. \\
	Une fois les tokens et le serveur mis en place il sera nécessaire de mettre tous les éléments en relation afin de vérifier le bon fonctionnement des communication. Ces taches sont en fait des test. Leurs charge est évaluée à 49 jours-hommes découpés en 21 jours-hommes pour les tokens UNIX et 28 jours-hommes pour les tokens Android.\\
	Enfin l'étape des tests finaux est évaluée à 42 jours-hommes.\\
	Ces informations sont résumées dans le tableau suivant :
	\newline

	\begin{tabular}{|l|l|l|}
	 \hline
	Tache & Durée & Ressources mobilisées \\ \hline
	État de l'art & 60 Jours & 7 personnes \\
	Outils pour le développement en C & 21 Jours & 4 personnes \\
	Outils pour le développement en Java & 21 Jours & 3 personnes \\
	Algorithmes pour le calcul des OTP en C & 30 Jours & 4 personnes \\
	Algorithmes pour le calcul des OTP en Java & 30 Jours & 3 personnes \\
	Client / serveur d'authentification & 45 Jours & 2 personnes \\
	Tokens en C & 30 Jours & 2 personnes \\
	Tokens en Java & 30 Jours & 3 personnes \\
	Mise en relation des éléments UNIX & 7 Jours & 3 personnes \\
	Mise en relation des éléments JAVA & 7 Jours & 4 personnes \\
	Tests finaux & 6 Jours & 7 personnes \\ \hline
	\end{tabular}


\newpage
%---------------------------------------------------------------------------------------------------------------------------------------------
\section{Planning général}
	\begin{figure}[h]
	\includegraphics[scale=0.33]{ressources.png}
	\caption{Répartition des ressources sur le projet.}
	\end{figure}
	\begin{figure}[h]
	\includegraphics[scale=0.33]{gantt.png}
	\caption{Planning des étapes du projet.}
	\end{figure}

	\begin{figure}[h]
	\includegraphics[scale=0.33]{pert.png}
	\caption{Ordonnancement des tâches du projet.}
	\end{figure}

	\newpage
%---------------------------------------------------------------------------------------------------------------------------------------------
\section{Procédés de gestion}
	\subsection{Gestion de la documentation}
		\begin{tabular}{|l|l|l|}
			\hline
			Document & Rédaction & Relecture \\
			\hline
			Documentation Bibliothèques C & Damien PICARD & Maxime MICHOTTE\\
			Documentation Bibliothèques Java & Yves ADEGOLOYE & Gaëtan FERRY\\
			Documentation OTP C & Damien PICARD & Claire HARDOUIN\\
			Documentation OTP Java & Gaëtan FERRY & Benjamin ZIGH \\
			Documentation Token C & Adrien SMONDACK & Maxime MICHOTTE\\
			Documentation Token Android & Yves ADEGOLOYE & Benjamin ZIGH\\
			Documentation Client\&Serveur & Claire HARDOUIN & Damien PICARD \\
			Rapports de tests & Gaëtan FERRY & Claire HARDOUIN \\
			Manuel d'utilisation & Benjamin ZIGH & Maxime MICHOTTE \\
			Rapport de projet & Maxime MICHOTTE & Benjamin ZIGH \\
			\hline 
		\end{tabular}

	\subsection{Gestion des configurations}
		\subsubsection{Gestion des versions}
			% \includegraphics[scale=1]{git-logo.png}

			Nous avons choisi d'utiliser le logiciel git pour gérer les différentes versions du projet et des documents). A l'aide de celui-ci, nous pourrons aisément contrôler les étapes d'avancement du projet au cours de son développement. 

			Git permet de développer collaborativement à distance en enregistrant et/ou fusionnant les modifications apportées par les différents membres de l'équipe. Nous ne nous attarderons pas ici sur le fonctionnement de git (voir \href{http://git-scm.com/book/fr/Démarrage-rapide-Rudiments-de-Git}{http://git-scm.com/book/fr/Démarrage-rapide-Rudiments-de-Git}).

			La majorité de l'équipe était déjà formée à l'utilisation du logiciel, nous avons donc pu être opérationnels rapidement pour la rédaction des divers documents techniques.

			La gestion du git est attribuée au responsable technique.

		\subsubsection{Hébergement}
			% \includegraphics[scale=0.5]{github-logo.png}

			Github est un site d'hébergement de dépôts git proposant l'hébergement gratuit des projets publics. Il permet également aux étudiants de créer des projets privés dans un cadre universitaire.

			Le site est bien établi dans la communauté open-source et garantit une disponibilité de 99.9\%. Le fonctionnement de git nous permet en plus d'avoir chacun des versions locales, donc la perte de données ne représente pas un risque important.


			Encore une fois, la majorité des membres de l'équipe était déjà inscrite sur github et il semblait logique de l'utiliser afin d'être efficace rapidement. 

			Github propose de nombreuses fonctionnalités supplémentaires, à la disposition du responsable technique pour communiquer avec l'équipe sur le développement du projet.

	\newpage
%---------------------------------------------------------------------------------------------------------------------------------------------
\section{Revues et points clef}
Points clés du projet :
\begin{itemize}
	\item Réalisation de l'état de l'art
	\begin{description}
		\item[Objectifs :] Prendre connaissance des protocles existant et de leurs spécificités, puis déterminer ceux que nous utiliserons.
		\item[Prévisions :] Du 22/11/13 au 20/01/14.
		\item[Objet des vérifications :] Accord du client.
		\item[Modalité de contrôle :] Rapport détaillé sur chaque protocoles étudiés ainsi que nos conclusions.
		\item[Intervenants :] Toute l'équipe.
	\end{description}
	\item Création des outils de développement
	\begin{description}
		\item[Objectifs :] Fournir des bibliothèques de fonctions nécessaires à la réalisation des étapes suivantes
		\item[Prévisions :] Du 21/01/14 au 11/02/14 - \'Etat de l'art terminé.
		\item[Objet des vérifications :] Compilation du code + Tests.
		\item[Modalité de contrôle :] Documentation à remettre au client.
		\item[Intervenants :] Par groupes.
	\end{description}
	\item Implémentation des algorithmes de calculs des OTP
	\begin{description}
		\item[Objectifs :] Concevoir des outils de calculs et de vérification d'OTP.
		\item[Prévisions :] Du 11/02/14 au 13/03/14 - Outils de développement disponibles.
		\item[Objet des vérifications :] Compilation du code + tests.
		\item[Modalité de contrôle :] Documentation à remettre au client.
		\item[Intervenants :] Par groupes.
	\end{description}
	\item Création des tokens
	\begin{description}
		\item[Objectifs :] Fournir des générateurs d'OTP.
		\item[Prévisions :] Du 13/03/14 au 12/04/14 - Implémentation des algorithmes terminée.
		\item[Objet des vérifications :] Retour d'un OTP respectant les spécifications suite à un évènement déclencheur.
		\item[Modalité de contrôle :] Manuel d'utilisation à remettre au client.
		\item[Intervenants :] Par groupes.
	\end{description}
	\item Mise en relation des éléments
	\begin{description}
		\item[Objectifs :] Faire fonctionner les outils développés précédemment en créant le serveur d'authentification et le client.
		\item[Prévisions :] Du 27/04/14 au 04/05/14 - Tokens disponibles.
		\item[Objet des vérifications :] Authentification avec succès.
		\item[Modalité de contrôle :] Documentation à remettre au client.
		\item[Intervenants :] Par groupes.
	\end{description}
	\item Test finaux
	\begin{description}
		\item[Objectifs :] Assurer une sécurité mi-forte (authentification à deux facteurs).
		\item[Prévisions :] Du 04/05/14 au 10/05/14 - Authentifications possible.
		\item[Objet des vérifications :] Résultat des tests positifs.
		\item[Modalité de contrôle :] Rapport de tests + Rapport de projet à remettre au client.
		\item[Intervenants :] Toute l'équipe.
	\end{description}
\end{itemize}
%---------------------------------------------------------------------------------------------------------------------------------------------
\section{Procédure de suivi et d'avancement}

	L'évolution du projet sera suivi en interne par : 
	\begin{itemize}
		\item des réunions régulières 
		\begin{description}
			\item[Objectif :] Faire le point sur l'avancement de chacun dans la tâche qui leur à été attribué et combler les lacunes qui pourraient se présenter.
			\item[Fréquence :] Toutes les deux semaines.
		\end{description}
		\item des séances de brainstorming par groupes
		\begin{description}
			\item[Objectif :] Décider des solutions à mettre en oeuvre et de l'organisation du travail au sein du groupe pour résoudre la problématique.
			\item[Fréquence :] Aussi souvent que nécessaire.
		\end{description}
	\end{itemize}

	Côté client, l'avancement du projet pourra être suivi en permanence par l'intermédiaire du site d'hébergement Github. Ce procédé permettra aux clients de constater non seulement l'évolution de la documentation et des outils de développement, mais aussi d'avoir une idée plus précise sur la contribution de chaque groupes et de la régularité de leur travail.

	De plus, chaque étapes débouchera sur la rédaction de documentations (compte-rendu ou manuel d'utilisation) dont une version imprimée sera remise en mains propre aux clients par la MOA.


	Outre les soutenances prévues en Janvier et en Mai, la MOA prévois au minimum deux réunions par mois avec les clients pour faire le points sur l'avancement du projet et sur les besoins du client (pouvant entrainer une réactualisation de la documentation).

\end{document}
